% Document
\documentclass[12pt, a4paper]{article}
\usepackage[T1]{fontenc}
\usepackage[utf8]{inputenc}
\usepackage{authblk}
\usepackage{lipsum}

% Fig. and table formating
\usepackage{epsfig}
\usepackage{tabu}
\usepackage{rotating}
\usepackage{pbox}
\usepackage{framed, multicol}
\usepackage[framemethod=TikZ]{mdframed}
\usepackage{float}
\usepackage[left=1.25 in, right=1.25 in, top=1.25 in, bottom=1.25 in]{geometry}

\usepackage{caption}

% Fonts - Mathtime
%\usepackage{txfonts}
\usepackage{amsmath} % Add amssymb if not using Mathtime
\newcommand\numberthis{\addtocounter{equation}{1}\tag{\theequation}}

% Text
\setlength{\parindent}{0.5in}
\frenchspacing  \tolerance = 800  \hyphenpenalty = 800

\usepackage{lineno} % Line numbers
\def\linenumberfont{\normalfont\footnotesize\ttfamily}
\setlength\linenumbersep{0.2 in}

\usepackage{setspace}

% Format section and subsection headers
\makeatletter
\renewcommand{\section}{\@startsection
{section}%                   % the name
{1}%                         % the level
{0mm}%                       % the indent
{-\baselineskip}%            % the before skip
{0.5\baselineskip}%          % the after skip
{\normalfont\bf\large}} % the style

\renewcommand{\subsection}{\@startsection
{subsection}%                   % the name
{2}%                         % the level
{0mm}%                       % the indent
{-\baselineskip}%            % the before skip
{0.5\baselineskip}%          % the after skip
{\normalfont\bf}} % the style
\makeatother

% Other
\usepackage{graphicx}
\usepackage[singlelinecheck=false,font=small,labelfont=bf]{caption}
\usepackage[justification=centering]{caption}

% Bibliography
\usepackage[numbers, compress]{natbib} % Bibliography - APA
\bibpunct{(}{)}{;}{a}{}{,}

% Format the Bibliography appropriately
% increase \bibhang to take care of the numbers
\setlength{\bibhang}{0pc}
\makeatletter
%patch \@lbibitem to print the current number before the authors
\patchcmd{\@lbibitem}
  {]}
  {] \theNAT@ctr. \newline }
  {}{}
\makeatother


%%%%%% FRONT MATTER %%%%%%%%%

\begin{document}

\doublespacing

\linenumbers
\section*{Introduction}

Infectious agents can have major impacts on animal populations through changing
population dynamics and stability (Dobson 1992), altering predator prey
interactions (Joly 2004), and even causing species decline and extinction
(Viana 2015, Daszak 1999).  Accurate estimates of parasite-induced host
mortality (PIHM) in wild animals are important for understanding what regulates both
host and parasite populations and to make predictions about disease
transmission in natural systems. Although a negative impact on host fitness is
a fundamental component of parasitism (Lafferty and Kuris 2002, Poulin book,
other definition sources), it is notoriously difficult to quantify PIHM in wild animal populations (cite, McCallum 2008?, lester
1984).

To conclusively identify PIHM in wild animal populations, it is necessary to
experimentally infect and track host populations in ways that are rarely
possible in most host-parasite systems (citations).  Instead, parasitologists
are often only able to collect a certain number of hosts and determine each
host's parasite intensity.  This snapshot, distributional data is far from the
ideal type of data for addressing questions regarding PIHM, but the reality is
that this is the type of data on which most questions regarding PIHM are asked
(citations). The two primary questions that one may wish to ask given a snapshot host-
parasite dataset are: 1) Is PIHM occurring in this system? and 2) How does
host survival change as parasite intensity increases?  MORE

The first of these two questions was addressed by \cite{Crofton1971a} developed
a method to test for PIHM using the truncation of the negative binomial
distribution.  In short, the Crofton Method assumes that the distribution of
parasites across hosts before mortality occurs follows a negative binomial
distribution \citep{AndersonandMay1978,Shaw1998}.  As heavily infected hosts
begin to die, the negative binomial distribution gets truncated because heavily
infected hosts dies and are no longer observed in a sample. In other words, the
observed host-parasite distribution and the pre-mortality host-parasite
distribution will predict substantially different numbers of of highly infected
hosts (because those have died due to infection) but similar number of hosts with low infection
loads (because those have survived).  \citep{Crofton1971a} noted that by
starting with all the observed data and iteratively fitting a negative binomial
distribution to hosts with lower and lower parasite loads, one could determine
whether or not PIHM was occurring in the system (Figure XA).  This was done
graphically by determining whether parameters of the truncated negative binomial distributions showed a substantial change as hosts with smaller and smaller parasite intensities were fitted (Figure XB).  The Crofton Method and this graphical technique for determining whether or not PIHM is occurring are both still used \citep{Ferguson2011}. A more thorough description and implementation of the Crofton Method is given in Appendix X.

The second question regarding PIHM is how host survival changes as parasite
intensity increases. The Crofton Method on its own cannot answer this question
and \cite{Adjei1986} proposed a method to determine how host survival
probability changes with increasing parasite-load.  The Adjei Method proceeds
by first using the Crofton Method to estimate the pre-mortality parameters for
a host-parasite distribution (describe what these are) and then, given these parameters, estimates a host-survival function that describes how the probability of host-survival changes with increasing parasite load (see Appendix X for a full description).  From this host-survival function, the Adjei Method can estimate important host-parasite quantities such as the parasite intensity at which 50\% of hosts succumb to PIHM ($LD_{50}$), as well as the percent of hosts in a population succumbing to PIHM \citep{Adjei1986}.

Asking questions regardin


\section*{Methods}

\subsection*{The likelihood method for estimating PIHM}

Given the potential deficiencies of the Adjei Method, we provide an alternative
approach for estimating parasite-induced host mortality (PIHM) that makes less
assumption than the previously proposed Adjei Method.  The likelihood method
does not require any binning or alteration of the data, potentially reduces the
number of parameters that need to be estimated, and allows for standard
statistical techniques to be used to assess the significance of PIHM in a
system.

The likelihood method makes the following assumption about the host-parasite
system. First, as with all previously proposed methods for estimating PIHM, the
likelihood method assumes that the pre-mortality distribution follows a
negative binomial distribution ($g(x)$) with parameter $\mu_p$ and $k_p$. The
validity of the assumption is an inherent problem with all the PIHM methods
proposed to date and we address this thoroughly in the Discussion. The second
assumption that the likelihood method makes is that the host survival function
takes the form of a logistic curve given by

\begin{equation}
    h(x | a, b) = h_x = \dfrac{e^{a + b \log(x)}}{1 + e^{a + b \log(x)}}
    \label{eq:logistic}
\end{equation}

where $x$ is the parasite intensity in a given host and $a$ and $b$ are the two parameters of the function. Generally, a larger $a$ allows for hosts to tolerate larger parasite loads before experiencing parasite-induced mortality and a more negative $b$ leads to a more rapid decline in the probability of host survival as parasite intensity increases.  The value $exp(a / |b|)$ (notation) is typically referred to as the $LD_50$, which gives the parasite intensity at which 50\% of host experience mortality.  With these two explicit assumptions, the likelihood method tries to estimate the 4 parameters $\mu_p$, $k_p$, $a$, and $b$.

To estimate these parameters, we need to define a probability distribution that gives the probability of having a parasite load of $x$ parasites conditional on host survival.  Using the standard rules of conditional probability  This distribution can be written as

\begin{equation}
    P(x | \text{survival}) = \dfrac{P(\text{survival} | x) * P(x)}{P(\text{survival})}
    \label{eq:concept}
\end{equation}

One can now see that $P(\text{survival} | x)$ is the survival function $h(x; a, b)$, $P(x)$ is the pre-mortality parasite distribution $g(x; \mu_p, k_p)$ and $P(survival) = \sum_0^{\infty} P(\text{survival} | x) * P(x) =  \sum_{x=0}^{\infty} h(x; a, b)  * g(x; \mu_p, k_p)$. Therefore equation \ref{eq:concept} can be written as

\begin{equation}
    P(x | survival) = \dfrac{h(x; a, b)  * g(x; \mu_p, k_p)}{\sum_{x=0}^{\infty} h(x; a, b)  * g(x; \mu_p, k_p)}
    \label{eq:dist}
\end{equation}

Using this probability distribution, one can then find the parameters $\mu_p$, $k_p$, $a$, $b$ that maximize the likelihood of an observed host-parasite dataset $\mathbf{x}$.

Alternatively, one could apply the Crofton Method to estimate $\mu_p$ and $k_p$ and then find the maximum likelihood estimates of $a$ and $b$ and the corresponding $LD_{50}$.  A final option would be to follow the example of \citep{Ferguson2011} and assume $k_p = 1$ and only estimate $a$, $b$ and $\mu_p$.

\subsection{Testing the ability of approaches to identify PIHM}

To test the ability of the Adjei Method and the Likelihood Method to identify whether or not PIHM was occurring in a system, we randomly generated data using the following procedure.  First, we drew $N_p$ randomly infected hosts from a
negative binomial distribution with parameters $\mu_p$ and $k_p$.  This represented the dataset observed before mortality. Second, we chose values of $a$ and $b$ and calculated the probability of survival
for all $N_p$ hosts using equation \ref{eq:logistic}.  Third, we drew $N_p$ random numbers from a uniform distribution
between 0 and 1 and if host survival probability was less than this random
number, the host experienced parasite-induced mortality.  The surviving
hosts comprised the dataset ($\mathbf{x}$) that would be obtained in the field, after PIHM.

Using both the pre-mortality and post-mortality simulated datasets,  we assumed
that the values of $N_p$, $\mu_p$, and $k_p$ were known and tested the ability of both methods to correctly determine whether or not PIHM was occurring.  While this scenario is unrealistic because the parameters $N_p$,
$\mu_p$, and $k_p$ are always unknown, we implemented this scenario as a baseline to
establish the efficacy of the methods independent of the estimates of $N_p$, $\mu_p$ and $k_p$.  For the Adjei Method, $N_p$, $\mu_p$, and $k_p$ are estimated using the Crofton Method, while $\mu_p$ and $k_p$ in the likelihood method can be estimated jointly with $a$ and $b$ or via the Crofton Method.   If a
method could not correctly predict whether or not PIHM was occurring under these idealistic conditions, we considered this strong evidence of the unreliability of this method.

We used three different values of $\mu_p$ (10, 100, 500) and for each $\mu_p$ we examined three different survival functions that had graduate, moderate, and sharp decreases in host survival with increasing parasite intensity.  For a given $\mu_p$, each survival function had the same $LD_{50}$, but different values of $a$ and $b$ (Table X).  We examined each $\mu_p$-survival function pair at  three levels of parasite
aggregation, $k_p = 0.1, 0.5, 1$, which are realistic values of parasite aggregation in natural populations \citep{Shaw1998}.  For each of these parameter
combinations we simulated 150 datasets and tested the probability of each method incorrectly identifying PIHM in the pre-mortality dataset (Type I error) and incorrectly failing to identify PIHM in the post-mortality dataset (Type II error).  For each method, we we used likelihood ratio test to determine whether the full model with PIHM provided a significantly better fit than the reduced model without PIHM at significance level 0.05 (Appendix X).  We tested all each parameter combinations for pre-mortality population sizes of $N_p$ = [300, 500, 1000, 2000, 5000, 7500,
10000].

\subsection*{Testing ability of PIHM approaches to recover survival function}

To compare the ability of the Adjei Method and the likelihood method to recover $LD_{50}$ and the parameters $a$ and $b$ or the survival function, we used the same simulation procedure and parameter combinations described above.For each parameter
combination we simulated 150 datasets, estimated $a$, $b$, and $LD_{50}$ and calculated the standardized bias and
precision \citep{Walther2005} for these estimates over varying pre-mortality host population sizes  $N_p$ = [300, 500, 1000, 2000, 5000, 7500,
10000]. $N_p$ is not technically the sample size on which the methods are being
tested because parasite-induced mortality reduces $N_p$ for each simulated
dataset.  We therefore used the average number of surviving hosts over all 150 simulations for a given parameter combination as our measure of sample size.

\subsection*{Application to data}

We tested the likelihood method on the datasets given in \citep{Crofton1971a} and \citep{Adjei1986} which both papers reported seeing PIHM in the respective populations.

\section*{Results}

\subsection*{Detecting PIHM}

\subsection*{Recovery of the survival function}

\subsection*{Application to data}


\section*{Discussion}

Parasite-induced host mortality is of substantial interest in many systems, but determining whether it is occurring given only observational data is notoriously difficult.  Many of the previous methods derived to determine the effect of PIHM on a host-parasite system limit their inference to answering the yes or no question of whether or not PIHM is occurring.  While a relevant question, it is often of interest to know something about the host survival function which can provide information regarding important properties of the host-parasite system, such as $LD_{50}$ and percent of the population suffering PIHM.

We show that the Adjei Method, the only currently proposed method to estimate the host survival function and the $LD_{50}$ from observational PIHM, has some serious methodological problems that result in biased estimates of the host survival function even under the most idealistic conditions.  Interestingly, despite these flaws, the Adjei Method can still produce unbiased and precise estimates of the $LD_{50}$ when host-parasite systems show aggregation close to $k = 1$

To attempt to ameliorate the flaws in the Adjei Method, we proposed a more
general method to determine both whether or not PIHM is occurring in a system
and the shape of the survival function.  We show that this method is
asymptotically unbiased when estimating the host-survival function for all of
the parameter space that we explored, but can produce seriously biased
estimates of the host survival function for sample sizes typically observed in
many host-parasite studies.  However, we found that the likelihood method
produces unbiased and precise estimates of the $LD_{50}$ for small, realistic
sample sizes.

Moreover, the likelihood method...

% Discuss all of the problems with these methods

What do these findings tell us about our ability to go beyond saying whether or
not PIHM is occurring in a system?  While we have generalized and improved upon
the previously existing methods for estimating PIHM, we cannot belie the fact
that estimating the host survival function from observational data alone is
ladened with assumptions and difficulties.

The most fundamental
assumption of all methods for estimating PIHM is that the shape of the pre-
mortality host-parasite distribution is known. In the discrete case, this
distribution is negative binomial, while in the continuous case the distribution can be gamma of exponential (\citep{Ferguson2011}).  While there is substantial empirical and
theoretical evidence to justify the use of the negative binomial distribution
as the pre-mortality distribution for macroparasites across hosts (Crofton, Shaw, Anderson and Gordon, etc), it is widely recognized that different processes can lead to a variety of distributions of parasites across hosts (Wilber, Duerr etc).  However, the critical assumption of the pre-mortality distribution is not that the processes leading the pre-mortality distribution generate a negative binomial distribution, but rather that the pre-mortality distribution is well-fit by a negative binomial. The extreme flexibility of the negative binomial distribution in the discrete case of the gamma distribution in the continuous case make them reasonable candidate distributions for the pre-mortality distributions.  Therefore, we do not see this assumption as central problem in any of the proposed methods.

However, to use the pre-mortality distribution to infer whether or not the PIHM
is occurring in a system requires an explicit assumption about the host
survival function and the shape of the post-mortality distribution.  Regarding
the host-survival function, all methods of PIHM assume that the host-survival
function is such that uninfected individuals and individuals with low parasite
intensity experience essentially no PIHM.  \cite{Lanciani1989} illustrated this
by showing that when hosts experienced a linear decrease in survival
probability the Crofton Method could not detect PIHM.  This result, relates to
the shape of the post-mortality distribution.  Given that the pre-mortality
distribution is well-fit by a negative binomial, a linear host-survival
function will result in a post-mortality distribution that is also well-fit by
a negative binomial distribution (why?).  In this case, one would be unable to
identify PIHM because the pre-mortality distribution



As parasitologist, we often want to get the most out of our





\singlespacing
\bibliographystyle{/Users/mqwilber/Dropbox/Documents/Bibformats/ecology_letters.bst}
\bibliography{/Users/mqwilber/Dropbox/Documents/Bibfiles/Projects_and_Permits-parasite_host_mortality}


\end{document}

