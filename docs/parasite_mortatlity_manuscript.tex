% Document
\documentclass[12pt, a4paper]{article}
\usepackage[T1]{fontenc}
\usepackage[utf8]{inputenc}
\usepackage{authblk}
\usepackage{lipsum}

% Fig. and table formating
\usepackage{epsfig}
\usepackage{tabu}
\usepackage{rotating}
\usepackage{pbox}
\usepackage{framed, multicol}
\usepackage[framemethod=TikZ]{mdframed}
\usepackage{float}
\usepackage[left=1.25 in, right=1.25 in, top=1.25 in, bottom=1.25 in]{geometry}

\usepackage{caption}

% Fonts - Mathtime
%\usepackage{txfonts}
\usepackage{amsmath} % Add amssymb if not using Mathtime
\newcommand\numberthis{\addtocounter{equation}{1}\tag{\theequation}}

% Text
\setlength{\parindent}{0.5in}
\frenchspacing  \tolerance = 800  \hyphenpenalty = 800

\usepackage{lineno} % Line numbers
\def\linenumberfont{\normalfont\footnotesize\ttfamily}
\setlength\linenumbersep{0.2 in}

\usepackage{setspace}

% Format section and subsection headers
\makeatletter
\renewcommand{\section}{\@startsection
{section}%                   % the name
{1}%                         % the level
{0mm}%                       % the indent
{-\baselineskip}%            % the before skip
{0.5\baselineskip}%          % the after skip
{\normalfont\bf\large}} % the style

\renewcommand{\subsection}{\@startsection
{subsection}%                   % the name
{2}%                         % the level
{0mm}%                       % the indent
{-\baselineskip}%            % the before skip
{0.5\baselineskip}%          % the after skip
{\normalfont\bf}} % the style
\makeatother

% Other
\usepackage{graphicx}
\usepackage[singlelinecheck=false,font=small,labelfont=bf]{caption}
\usepackage[justification=centering]{caption}

% Bibliography
\usepackage[numbers, compress]{natbib} % Bibliography - APA
\bibpunct{(}{)}{;}{a}{}{,}

% Format the Bibliography appropriately
% increase \bibhang to take care of the numbers
\setlength{\bibhang}{0pc}
\makeatletter
%patch \@lbibitem to print the current number before the authors
\patchcmd{\@lbibitem}
  {]}
  {] \theNAT@ctr. \newline }
  {}{}
\makeatother


%%%%%% FRONT MATTER %%%%%%%%%

\begin{document}

\doublespacing

\linenumbers
\section*{Introduction}

Infectious agents can have major impacts on animal populations through changing
population dynamics and stability (Dobson 1992), altering predator prey
interactions (Joly 2004), and even causing species decline and extinction
(Viana 2015, Daszak 1999).  Accurate estimates of parasite-induced host
mortality (PIHM) in wild animals are important for understanding what regulates both
host and parasite populations and to make predictions about disease
transmission in natural systems. Although a negative impact on host fitness is
a fundamental component of parasitism (Lafferty and Kuris 2002, Poulin book,
other definition sources), it is notoriously difficult to quantify PIHM in wild animal populations (cite, McCallum 2008?, lester
1984).

To conclusively identify PIHM in wild animal populations, it is necessary to
experimentally infect and track host populations in ways that are rarely
possible in most host-parasite systems \citep{McCallum2000a}.  Instead, parasitologists
are often only able to determine the parasite intensity on some number of sampled hosts.  This snapshot, distributional data is far from the
ideal type of data for addressing questions regarding PIHM, but the reality is
that this is the type of data on which most questions regarding PIHM are asked
(citations). The two primary questions that one may wish to ask given a snapshot host-
parasite dataset are: 1) Is PIHM occurring in this system? and 2) How does
host survival change as parasite intensity increases?  MORE

The first of these two questions was addressed by \cite{Crofton1971a} who developed
a method to test for PIHM using the truncation of the negative binomial
distribution.  In short, the Crofton Method assumes that the distribution of
parasites across hosts before mortality occurs follows a negative binomial
distribution \citep{AndersonandMay1978,Shaw1998}.  As heavily infected hosts
begin to die, the negative binomial distribution is truncated because heavily
infected hosts die and are no longer observed in a sample. In other words, the
observed host-parasite distribution and the pre-mortality host-parasite
distribution will predict substantially different numbers of of highly infected
hosts (because those have died due to infection), but similar number of hosts with low infection
loads (because those have survived).  \citep{Crofton1971a} noted that by
starting with all the observed data and iteratively fitting a negative binomial
distribution to hosts with lower and lower parasite loads, one could determine
whether or not PIHM was occurring in the system (Figure XA).  This was done
graphically by determining whether parameters of the truncated negative binomial distributions showed a substantial change as hosts with smaller and smaller parasite intensities were fitted (Figure XB).  The Crofton Method and this graphical technique for determining whether or not PIHM is occurring are both still used \citep{Ferguson2011}. A more thorough description and implementation of the Crofton Method is given in Appendix X.

The second question regarding PIHM is how host survival changes as parasite
intensity increases. The Crofton Method on its own cannot answer this question
and \cite{Adjei1986} proposed a method to determine how host survival
probability changes with increasing parasite load.  The Adjei Method proceeds
by first using the Crofton Method to estimate the pre-mortality parameters for
a host-parasite distribution (describe what these are) and then, given these
parameters, estimates a host-survival function that describes how the
probability of host-survival changes with increasing parasite load (see
Appendix X for a full description).  From this host-survival function, the
Adjei Method can estimate important host-parasite quantities such as the
parasite intensity at which 50\% of hosts succumb to PIHM ($LD_{50}$), as well
as the percent of hosts in a population succumbing to PIHM \citep{Adjei1986}. A more thorough description of the Adjei Method is given in Appendix X.

Despite these methods both being over two decades old, they are still the
primary means of answering questions about PIHM given distributional data
\citep[\cite{Royce1990} but see][for an alternative to the Crofton
Method]{Ferguson2011}. However, both methods have a few important limitations.
The Crofton Method (and the method recently proposed by \cite{Ferguson2011})
relies on a visual test to determine whether or not PIHM is occurring in a
system.  With no clear decision rule, it can be difficult to determine the
significance of PIHM in a given host-parasite system.  Moreover, previous
studies have shown that the Crofton Method can fail to detect PIHM depending on
the shape of the host survival function \citep{Lanciani1989}.

In theory, the Adjei Method can provide a means to assess the significance of
PIHM in a system as well as determine the shape of the host survival function.
In practice, however, the Adjei Method has never been thoroughly tested and
relies on a number of questionable data manipulations.  For example, in order
for the Adjei Method to be used to estimate PIHM, the observed data must be
altered; a less than ideal property for any statistical method.

In this study, we propose a novel method for estimating both whether or
not PIHM is occurring in a host-parasite system and the shape of the survival
function.  We compare the ability of our method to answer the two PIHM
questions given above to the results given by the Adjei Method.  We show that
our method is superior to the Adjei Method and that the Adjei Method has
serious deficiencies when trying to answer either of the above questions.

\section*{Methods}

\subsection*{The Adjei Method for estimating PIHM}



\subsection*{The Likelihood Method for estimating PIHM}

Given the potential deficiencies of the Adjei Method, we provide an alternative
approach for estimating parasite-induced host mortality (PIHM) that makes less
assumption than the previously proposed Adjei Method.  The likelihood method
does not require any binning or alteration of the data, potentially reduces the
number of parameters that need to be estimated, and allows for standard
statistical techniques to be used to assess the significance of PIHM in a
system.

The likelihood method makes the following assumption about the host-parasite
system. First, as with all previously proposed methods for estimating PIHM, the
likelihood method assumes that the pre-mortality distribution follows a
negative binomial distribution ($g(x)$) with parameter $\mu_p$ and $k_p$. The
validity of the assumption is an inherent problem with all the PIHM methods
proposed to date and we address this thoroughly in the Discussion. The second
assumption that the likelihood method makes is that the host survival function
takes the form of a logistic curve given by

\begin{equation}
    h(x | a, b) = h_x = \dfrac{e^{a + b \log(x)}}{1 + e^{a + b \log(x)}}
    \label{eq:logistic}
\end{equation}

where $x$ is the parasite intensity in a given host and $a$ and $b$ are the two parameters of the function. Generally, a larger $a$ allows for hosts to tolerate larger parasite loads before experiencing parasite-induced mortality and a more negative $b$ leads to a more rapid decline in the probability of host survival as parasite intensity increases.  The value $exp(a / |b|)$ (notation) is typically referred to as the $LD_50$, which gives the parasite intensity at which 50\% of host experience mortality.  With these two explicit assumptions, the likelihood method tries to estimate the 4 parameters $\mu_p$, $k_p$, $a$, and $b$.

To estimate these parameters, we need to define a probability distribution that gives the probability of having a parasite load of $x$ parasites conditional on host survival.  Using the standard rules of conditional probability this distribution can be written as

\begin{equation}
    P(x | \text{survival}) = \dfrac{P(\text{survival} | x) * P(x)}{P(\text{survival})}
    \label{eq:concept}
\end{equation}

One can now see that $P(\text{survival} | x)$ is the survival function $h(x; a, b)$, $P(x)$ is the pre-mortality parasite distribution $g(x; \mu_p, k_p)$ and $P(survival) = \sum_0^{\infty} P(\text{survival} | x) * P(x) =  \sum_{x=0}^{\infty} h(x; a, b)  * g(x; \mu_p, k_p)$. Therefore equation \ref{eq:concept} can be written as

\begin{equation}
    P(x | survival) = \dfrac{h(x; a, b)  * g(x; \mu_p, k_p)}{\sum_{x=0}^{\infty} h(x; a, b)  * g(x; \mu_p, k_p)}
    \label{eq:dist}
\end{equation}

Using this probability distribution, one can then find the parameters $\mu_p$, $k_p$, $a$, and $b$ that maximize the likelihood of an observed host-parasite dataset $\mathbf{x}$.

Alternatively, one could apply the Crofton Method to estimate $\mu_p$ and $k_p$ and then find the maximum likelihood estimates of $a$ and $b$ and the corresponding $LD_{50}$.  A final option would be to follow the example of \citep{Ferguson2011} and assume $k_p = 1$ and only estimate $a$, $b$ and $\mu_p$.

\subsection{Testing the ability of approaches to identify PIHM}

To test the ability of the Adjei Method and the Likelihood Method to identify whether or not PIHM was occurring in a system, we randomly generated data using the following procedure.  First, we drew $N_p$ randomly infected hosts from a
negative binomial distribution with parameters $\mu_p$ and $k_p$.  This represented the dataset observed before mortality. Second, we chose values of $a$ and $b$ and calculated the probability of survival
for all $N_p$ hosts using equation \ref{eq:logistic}.  Third, we drew $N_p$ random numbers from a uniform distribution
between 0 and 1 and if host survival probability was less than this random
number, the host experienced parasite-induced mortality.  The surviving
hosts comprised the dataset ($\mathbf{x}$) that would be obtained in the field, after PIHM.

Using both the pre-mortality and post-mortality simulated datasets,  we assumed
that the values of $N_p$, $\mu_p$, and $k_p$ were known and tested the ability of both methods to correctly determine whether or not PIHM was occurring.  While this scenario is unrealistic because the parameters $N_p$,
$\mu_p$, and $k_p$ are always unknown, we implemented this scenario as a baseline to
establish the efficacy of the methods independent of the estimates of $N_p$, $\mu_p$ and $k_p$.  For the Adjei Method, $N_p$, $\mu_p$, and $k_p$ are estimated using the Crofton Method, while $\mu_p$ and $k_p$ in the likelihood method can be estimated jointly with $a$ and $b$ or via the Crofton Method.   If a
method could not correctly predict whether or not PIHM was occurring under these idealistic conditions, we considered this strong evidence of the unreliability of this method.

We used three different values of $\mu_p$ (10, 100, 500) and for each $\mu_p$ we examined three different survival functions that had graduate, moderate, and sharp decreases in host survival with increasing parasite intensity.  For a given $\mu_p$, each survival function had the same $LD_{50}$, but different values of $a$ and $b$ (Table X).  We examined each $\mu_p$-survival function pair at  three levels of parasite
aggregation, $k_p = 0.1, 0.5, 1$, which are realistic values of parasite aggregation in natural populations \citep{Shaw1998}.  For each of these parameter
combinations we simulated 150 datasets and tested the probability of each method incorrectly identifying PIHM in the pre-mortality dataset (Type I error) and incorrectly failing to identify PIHM in the post-mortality dataset (Type II error).  For each method, we we used likelihood ratio test to determine whether the full model with PIHM provided a significantly better fit than the reduced model without PIHM at significance level 0.05 (Appendix X).  We tested all each parameter combinations for pre-mortality population sizes of $N_p$ = [300, 500, 1000, 2000, 5000, 7500,
10000].

\subsection*{Testing ability of PIHM approaches to recover survival function}

To compare the ability of the Adjei Method and the likelihood method to recover $LD_{50}$ and the parameters $a$ and $b$ or the survival function, we used the same simulation procedure and parameter combinations described above.For each parameter
combination we simulated 150 datasets, estimated $a$, $b$, and $LD_{50}$ and calculated the standardized bias and
precision \citep{Walther2005} for these estimates over varying pre-mortality host population sizes  $N_p$ = [300, 500, 1000, 2000, 5000, 7500,
10000]. $N_p$ is not technically the sample size on which the methods are being
tested because parasite-induced mortality reduces $N_p$ for each simulated
dataset.  We therefore used the average number of surviving hosts over all 150 simulations for a given parameter combination as our measure of sample size.

\subsection*{Application to data}

We tested the ability of the Adjei Method and the Likelihood Method to identify
PIHM on 6 host-parasite datasets given in \cite{Crofton1971a} and 4 datasets
given in \cite{Adjei1986}. In the \cite{Crofton1971a} datasets, the host was
the snail \emph{Gammarus pulex} which acts as the intermediate host for the
acanthocephalan \emph{Polmorphus minutus}. In the \cite{Adjei1986} datasets,
the hosts were two species of lizard fish \emph{Saurida tumbil} and
\emph{Saurida undosquamis} that were infected by the cestode parasite
\emph{Callitetrarhynchus gracilis}.  Males and females of both fish species
were considered separately.

In both these studies, the authors reported PIHM and we test whether the Adjei
Method and or the Likelihood Method also predict PIHM. For the 6 datasets from
\cite{Crofton1971a}, we truncated the data at 4 parasites, applied the Crofton
Method to estimate the pre-mortality distribution, and then ran the Likelihood
Method and Adjei Method using these pre-mortality parameters.  For the
\cite{Adjei1986} datasets, we followed the same procedure as the authors and
first truncated the data at 2 parasites and then fit the Crofton Method for the
female fish of both species.  We then parameterized the male pre-mortality
distributions for each species with the results from the females.  Finally, we
applied the Adjei Method and the Likelihood Method to determine whether or not
PIHM was significant for these species and compared our results to those given by the authors.

\section*{Results}

\subsection*{Detecting PIHM}



\subsection*{Recovery of the survival function}

The Likelihood Method gave asymptotically unbiased estimates of the $LD_{50}$ for all combinations of parameters examined in this study (Figure X).  Even small sample sizes (< 500 hosts), the Likelihood Methods estimate of $LD_{50}$ was largely unbiased, with small biases occurring for host survival functions that were gradual (i.e. $\mu_p = 10, a = 5, b = -2.5; \mu_p = 100, a = 5, b= -1.5; \mu_p = 500, a = 10, b = -1.8$)   On the other hand, the Adjei Method always produced more biased estimates of the $LD_{50}$ than the Likelihood Method across all parameter combinations.  For $\mu_p = 10$, the $LD_{50}$ estimates from the Adjei Method were largely unbiased for large samples sizes, but showed increasing bias as sample size decreased, particularly for steep survival functions ($a = 20, b = -10$).  As $\mu_p$ increased, the Adjei method produced biased estimates of $LD_{50}$ across all sample sizes, with bias increasing as sample size decreased.

The precision of the $LD_{50}$ estimates for the Likelihood Method decreased (increasing coefficient of variation) as sample size decreased for all parameter combinations we examined.  The $LD_{50}$ estimates from the Adjei Method showed a similar pattern, with large decreases in precision occurring for the steepest survival function across all values of $\mu_p$ (Figure X).

In terms of the host survival function, the Likelihood Method gave
asymptotically unbiased estimates of $a$ and $b$ as sample size increased for
all parameter combinations considered.  However, as sample size decreased, the
likelihood method tended to produce severely biased estimates of $a$ and $b$.
This was generally more pronounced for steeper survival functions and more
aggregated pre-mortality distributions (Figure X).  The Adjei Method produced
biased estimates of $a$ and $b$ across all sample sizes, with the bias
consistently being larger when the survival function was steeper. The bias of
the Adjei Method also increased as $\mu_p$ increased.

\subsection*{Application to data}

Of the 10 datasets we considered, the previous authors claimed to detect PIHM
in 7 of them (Table \ref{table:pihm}).  The Likelihood Method parameterized
from the pre-mortality parameters of the Crofton Method detected significant
PIHM in 6 of these 7 datasets at a significance level of 0.05.  The only
dataset in which the the Likelihood Method parameterized from the Crofton
Method did not detect a significant effect of PIHM was the the Adjei dataset
for \emph{S. tumbil}.  For this case there was a marginally significant effect
of PIHM the likelihood ratio test giving $\chi^2_{df=2} = 5.34; p = 0.069$.

The Adjei Method detected significant PIHM in 9 of the 10 datasets given (Table \ref{table:pihm}).  This is consistent with the the previous results which show that the Adjei Method has a very high Type I error rate.  The Likelihood Method in which all parameters were jointly estimated only predicted significant PIHM in 4 of the 10 datasets.


Discuss LD50 values?


\section*{Discussion}

Parasite-induced host mortality is of substantial interest in many systems, but
determining whether it is occurring given only observational data is
notoriously difficult.  We show that the Adjei Method, the only currently
proposed method to estimate the host survival function and the $LD_{50}$ from
observational PIHM, has some serious methodological problems that result in
biased estimates of the host survival function and the $LD_{50}$ even under the most idealistic conditions.  Moreover, we show that the Adjei Method has a seriously inflated Type I error rate, meaning it will often detect PIHM even when it is not present.

To attempt to ameliorate the flaws in the Adjei Method, we proposed a more
general method to determine both whether or not PIHM is occurring in a system
and the shape of the survival function.  We show that this method is
asymptotically unbiased when estimating the host-survival function for all of
the parameter space that we explored and we found that it
produces unbiased and precise estimates of the $LD_{50}$ for small, realistic
sample sizes.  Moreover, this novel method has reasonable type I and type II error rates when attempting to predict whether or not PIHM is occurring in a system.  However, we note that the Likelihood Method produces seriously biased
estimates of the host survival function ($a$ and $b$) for sample sizes typically observed in many host-parasite studies and should not be used to estimate the exact shape of the host survival function.

We fit both the Likelihood Method and the Adjei Method to empirical data to
determine whether they could detect PIHM that had been previously reported
based on visual assessments.  Consistent with our simulation results, we found
that the Adjei Method tended to detect PIHM where it had not been previously
reported, while the Likelihood Method's detection of PIHM was consistent with
previously PIHM in a given dataset. Taken together, these results suggest that the Adjei Method is a fundamentally
flawed method for detecting PIHM and describing features of the host survival
function and we recommend using the Likelihood Method for detecting PIHM and describing attributes of the host survival function.

% Discuss all of the problems with these methods

While we improved upon
the previously existing methods for answering questions about PIHM, we cannot belie the fact
that estimating PIHM from observational data alone is
ladened with assumptions and difficulties \citep{McCallum2000a}. The most fundamental
assumption of all methods for estimating PIHM is that the shape of the pre-
mortality host-parasite distribution is known. In the discrete case, this
distribution is negative binomial, while in the continuous case the distribution can be gamma or exponential (\citep{Ferguson2011}).  While there is substantial empirical and
theoretical evidence to justify the use of the negative binomial distribution
as the pre-mortality distribution for macroparasites across hosts \citep{Calabrese2011,Anderson1982a,Shaw1998}, it is widely recognized that different processes can lead to a variety of distributions of parasites across hosts (Wilber, Duerr etc).  However, the critical assumption of the pre-mortality distribution is not that the processes leading the pre-mortality distribution generate a negative binomial distribution, but rather that the pre-mortality distribution is well-fit by a negative binomial. The extreme flexibility of the negative binomial distribution makes it a reasonable candidate distribution for the pre-mortality distributions.  Therefore, we do not see this assumption as central problem in any of the proposed methods.

However, to use the pre-mortality distribution to infer whether or not the PIHM
is occurring in a system requires an explicit assumption about the host
survival function and the shape of the post-mortality distribution.  Regarding
the host-survival function, all currently proposed methods of PIHM assume that the host-survival
function is such that uninfected individuals and individuals with low parasite
intensity experience essentially no PIHM.  \cite{Lanciani1989} illustrated this
by showing that when hosts experienced a linear decrease in survival
probability the Crofton Method could not detect PIHM.  As the most fundamental models of host-parasite dynamics predict a linear decrease in host survival probability with increasing host load \citep{AndersonandMay1978}, the failure of these methods to detect this relationship is a significant disconnect between empirical and theoretical disease ecology.

Related to the above result, all of these methods require that the post-mortality distribution be significantly different from a negative binomial distribution.  This is necessary because none of the above methods will be able to detect PIHM if a negative binomial distribution is an adequate fit to the post-mortality distribution [describe why?].  As many observed host-parasite distributions are not significantly different from a negative binomial distribution, there are limited cases where these methods can even be considered.

Finally, all of these methods assume that the truncation of a negative binomial
distribution is due to PIHM, but previous studies have shownthat a variety of other processes can lead to the truncation of a negative
binomial distribution such as within host parasite density-dependence, age-
dependent variation in host resistance and rate of infection
\citep{McCallum2000a,Anderson1982a,Rousset1996}.  Therefore, even detecting ``significant'' PIHM in a
dataset does not mean that PIHM is cause of the truncation.

Given these numerous caveats, is there a place in parasitology for methods that
estimate PIHM from distributional data?  We are in agreement with
\cite{Lester1984} that, at the very least, methods for estimating PIHM can provide preliminary insight into whether or not PIHM is worth further exploration.  However, we stress that these methods should only be used as an
exploratory tool when assessing the role of PIHM in a system and potential
users should critically evaluate whether they think they have a large enough
sample sizes and appropriate host survival functions for the methods proposed
in this paper to be applicable.  Even if they are applicable, inferring PIHM
from distributional data is no substitute for field or laboratory experiments and an in
depth understanding of the natural history of the host-parasite system under consideration.


\singlespacing
\bibliographystyle{/Users/mqwilber/Dropbox/Documents/Bibformats/ecology_letters.bst}
\bibliography{/Users/mqwilber/Dropbox/Documents/Bibfiles/Projects_and_Permits-parasite_host_mortality}

\renewcommand{\arraystretch}{1.2}
\begin{sidewaystable}

    \caption{Comparison of the PIHM predictions of previously used host-parasite datasets to those given by the Adjei Method and the Likelihood Method. The first column specifies the identity of the dataset, the second column specifies whether or not the author indicated that PIHM was occurring in the system, the third column indicates whether or not the Likelihood Method with parameters from the Crofton Method detects significant PIHM, the third column indicates whether the Likelihood Method with jointly estimated pre-mortality parameters detects PIHM, and the final column indicates whether the Adjei Method with pre-mortality parameters estimated from the Crofton Method detects PIHM.  If a method detects significant PIHM the $LD_{50}$ is given is parentheses}

    \centering
    \begin{tabular}{l l p{3cm} p{3cm} l}

    \hline\hline
    Data Set & Author detected PIHM? & Likelihood Method w/ Crofton Parameters & Likelihood Method w/out Crofton Parameters & Adjei Method \\

    \hline
    Crofton, Station 1 & Yes & Yes (7.27) & Yes (8.00) & Yes (9.33) \\
    Crofton, Station 2 & Yes & Yes (6.92) & No & Yes (14.95)\\
    Crofton, Station 3 & Yes & Yes (5.93) & Yes (7.04) & Yes (5.98) \\
    Crofton, Station 4 & No & No & No & Yes (7.99) \\
    Crofton, Station 5 & No & No & No & Yes (10.58) \\
    Crofton, Station 6 & No & No & No & No \\
    Adjei, \emph{S. tumbil} female & Yes (5.7) & No  & No   & Yes (6.37) \\
    Adjei, \emph{S. tumbil} male & Yes (3.4) & Yes (3.42) & Yes (3.28) & Yes (3.66)  \\
    Adjei, \emph{S. undosquamis} female & Yes (3.2) & Yes (3.04) & No  & Yes (3.11) \\
    Adjei, \emph{S. undosquamis} male & Yes (1.8) & Yes (1.83) & Yes (2.34) & Yes (1.78) \\


    \end{tabular}
    \label{table:pihm}


\end{sidewaystable}


\end{document}

