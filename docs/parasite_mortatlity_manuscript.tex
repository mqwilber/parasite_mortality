% Document
% Document
\documentclass[12pt, a4paper]{article}
\usepackage[T1]{fontenc}
\usepackage[utf8]{inputenc}
\usepackage{authblk}
\usepackage{lipsum}
\usepackage{tikz}

% Figure and table formating
\usepackage{epsfig}
\usepackage{tabu}
\usepackage{rotating}
\usepackage{pbox}
\usepackage{framed, multicol}
\usepackage[framemethod=TikZ]{mdframed}

% Set up Frame
\mdfdefinestyle{MyFrame}{%
    linecolor=black,
    skipabove=10pt,
    outerlinewidth=2pt,
    roundcorner=20pt,
    innertopmargin=10pt,
    innerbottommargin=10pt,%\baselineskip,
    innerrightmargin=20pt,
    innerleftmargin=20pt,
    backgroundcolor=gray!50!white}

\usepackage{float}
\usepackage[left=1 in, right=1 in, top=1.25 in, bottom=1.25 in]{geometry}

% Fonts - Mathtime
%\usepackage{txfonts}
\usepackage{amsmath} % Add amssymb if not using Mathtime

% Text
\setlength{\parindent}{0.5in}
\frenchspacing  \tolerance = 800  \hyphenpenalty = 800

\usepackage{lineno} % Line numbers
\def\linenumberfont{\normalfont\footnotesize\ttfamily}
\setlength\linenumbersep{0.1 in}

\usepackage{setspace}

% Format section and subsection headers
\makeatletter
\renewcommand{\section}{\@startsection
{section}%                   % the name
{1}%                         % the level
{0mm}%                       % the indent
{-\baselineskip}%            % the before skip
{0.5\baselineskip}%          % the after skip
{\normalfont\bf\large}} % the style

\renewcommand{\subsection}{\@startsection
{subsection}%                   % the name
{2}%                         % the level
{0mm}%                       % the indent
{-\baselineskip}%            % the before skip
{0.5\baselineskip}%          % the after skip
{\normalfont\bf}} % the style
\makeatother

% Other
\usepackage{graphicx}
\usepackage[singlelinecheck=false,font=small,labelfont=bf]{caption}

% Bibliography
\usepackage[numbers, compress]{natbib} % Bibliography - APA
\bibpunct{(}{)}{;}{a}{}{,}

% Format the Bibliography appropriately
% increase \bibhang to take care of the numbers
\setlength{\bibhang}{0pc}
\makeatletter
% patch \@lbibitem to print the current number before the authors
\patchcmd{\@lbibitem}
  {]}
  {] \theNAT@ctr. \newline }
  {}{}
\makeatother


%%%%%% FRONT MATTER %%%%%%%%%

\title{Methods for estimating parasite-induced mortality from intensity data and their limitations}
\author{Mark Wilber, Sara Weinstein, and Cherie Briggs}


\begin{document}


\maketitle

\section*{Abstract}

TODO

\doublespacing

\linenumbers
\section*{Introduction}

Infectious agents can have major impacts on animal populations through changing
population dynamics and stability \citep{Dobson1992}, altering predator-prey interactions \citep{Joly2004}, and
even causing species' decline and extinction \citep{DeCastro2005a,McCallum2012b}. Accurately estimating the impact
of these infectious agents in wildlife is critical to understanding what
regulates host and parasite populations, making predictions about disease
transmission, and managing disease outbreaks \citep{Langwig2015}. The impact of pathogens, such as rabies (Coyne et al. 1989) [get citation], bovine TB \citep{Cox2005}, and
rinderpest \citep{Tille1991}, is typically quantified based on the presence or absence of
disease, and does not account for the number of infectious agents present.
This method is sufficient for many bacterial and viral agents that reproduce within a
host, however for macroparasites pathology is linked to the intensity of infection and hosts cannot be simply categorized as infected and
uninfected \citep{AndersonandMay1979,Lafferty2002}.  Helminths exhibiting this intensity dependent pathology have
significant impacts on human health \citep{Brooker2004}, domestic livestock economics \citep{Roeber2013},
wildlife survival \citep{Kirk2003, Logiudice2003}.  While it is generally assumed that some fraction of
wild host populations must succumb to parasitic infections, it is notoriously
difficult to actually quantify parasite-induced host mortality (PIHM) in wild
animal populations \citep{McCallum2000a}.

Ideally, parasite-induced host mortality is
quantified by experimentally infecting and tracking individual hosts in the
wild population; however, for logistical and ethical reasons this method is
rarely feasible \citep{McCallum2000a}. Data on parasite intensity is much easier to collect and has
often been used to identify the presence of PIHM \citep{Crofton1971a,Lester1977,Lester1984,Lanciani1989,Royce1990,Ferguson2011} and quantify the
relationship between infection intensity and host mortality \citep{Adjei1986}.

\cite{Crofton1971a} first proposed that PIHM could be identified by comparing the
observed parasite distribution in the host population to the distribution
predicted in the absence of parasite-induced host mortality. We briefly introduce the Crofton Method here and provide a more detailed explanation of its implementation in \emph{Supplementary Material (SI)} 1. This method
assumes that, prior to host mortality, infection intensity in the host population follows a negative binomial distribution and the tail of the distribution is truncated as intensity dependent pathology removes the most heavily infected hosts. Assuming mortality occurs only in these heavily infected hosts, evidence of this parasite-induced mortality should then be detectable by iteratively
fitting a negative binomial distribution to hosts with lower and lower parasite
loads, and comparing the tail ends of these predicted distributions to the
observed parasite data.

The Crofton Method may be able to detect the presence of PIHM however, but it does not quantify the relationship between infection intensity and host survival
probability. \cite{Adjei1986} suggested that this relationship
could be calculated by first using the Crofton Method to estimate the pre-mortality parasite distribution and then using those distribution to calculate the
probability of host survival with increasing parasite intensity. To do this,
\cite{Adjei1986} modeled host survival as a logistic function and then used a generalized linear model (GLM) to estimate the logistic parameters (see \emph{SI} 2 for a technical description of the Adjei Method).
These methods appeared to provide an estimate for the parasite intensity at which
50\% of hosts exhibit PIHM ($LD_{50}$), as well as the unmeasurable fraction of the
population that was lost (\emph{SI} 2). However, to implement this method the observed data must be
modified to fit the GLM framework and subjectively binned when mean
infection intensity is high or sample sizes are small.

After 30 years, and despite clear limitations \citep{McCallum2000a}, these
methods (particularly the Crofton Method) are still discussed among
parasitologists and are the primary techniques for examining population level
impacts of parasitism using parasite intensity data. In these methods, PIHM can
only be identified by visually examining plots and, with no clear decision
rule, it can be difficult to determine the significance of PIHM across
different host-parasite systems. The survival function produced by the Adjei
Method offers one solution; however, this method requires manipulating the
original data and has never been tested.

Intensity data should only
be used to estimate parasite impacts on host populations if unbiased and accurate methods exist. In
this study, we first propose a novel method for calculating PIHM. We
next use simulations to compare our method with the previous Adjei
Method to test the ability of both methods to (1) detect occurrence of PIHM and (2) estimate the
lethal parasite load ($LD_{50}$) and the associated survival function.  We then
apply both methods to real datasets previously used in PIHM analyses and
compare the results. Finally, we discuss the limitations of inferring PIHM
from intensity data and whether any method for inferring PIHM has a place
in quantitative parasitology.

\section*{Methods}

\subsection*{A novel, likelihood-based method for estimating PIHM}

Here we propose a novel, likelihood-based method (henceforth Likelihood Method) that does not
require binning or data alteration, reduces the number of
parameters to be estimated, and  uses standard statistical techniques to determine PIHM significance. We provide Python code for implementing the Likelihood Method in \emph{SI} 4.

As with all previously proposed methods for estimating PIHM, the
Likelihood Method first assumes that prior to mortality the parasite distribution is described by a negative binomial $g(x; \mu_p, k_p)$, where $\mu_p$ and $k_p$ are the mean parasite intensity and aggregation before mortality, respectively (smaller $k_p$ indicates more aggregation).   Previous methods required calculating the total number of hosts before mortality ($N_p$) \citep{Crofton1971a,Adjei1986}, however this parameter is not needed in the Likelihood Method.

The Likelihood Method then assumes that the host survival function, which specifies the probability of a host surviving with $x$ parasites, follows the logistic curve given by

\begin{equation}
    h(x ; a, b) = \dfrac{e^{a - b \log(x)}}{1 + e^{a - b \log(x)}}
    \label{eq:logistic}
\end{equation}
With these two explicit assumptions, the Likelihood Method estimates four parameters: $\mu_p$, $k_p$, $a$, and $b$ by  first defining a probability distribution that gives the probability of having a parasite load of $x$ parasites conditional on host survival.  Using standard rules of conditional probability this distribution can be written as

\begin{equation}
    P(x | \text{survival}) = \dfrac{P(\text{survival} | x) * P(x)}{P(\text{survival})}
    \label{eq:concept}
\end{equation}

One can see that $P(\text{survival} | x)$ is the survival function $h(x; a, b)$, $P(x)$ is the pre-mortality parasite distribution $g(x; \mu_p, k_p)$ and $P(\text{survival}) = \sum_{x=0}^{\infty} P(\text{survival} | x) * P(x) =  \sum_{x=0}^{\infty} h(x; a, b)  * g(x; \mu_p, k_p)$. Therefore equation \ref{eq:concept} can be written as

\begin{equation}
    P(x | \text{survival}) = \dfrac{h(x; a, b)  * g(x; \mu_p, k_p)}{\sum_{x=0}^{\infty} h(x; a, b)  * g(x; \mu_p, k_p)}
    \label{eq:dist}
\end{equation}

Using this probability distribution, one can then find the parameters $\mu_p$, $k_p$, $a$, and $b$ that maximize the likelihood of an observed host-parasite dataset. The equation $\exp(a / b)$ can then be used to calculate the parasite $LD_{50}$, here defined as the infection intensity at which 50\% of hosts experience PIHM.

To estimate the significance of PIHM in a host-parasite system, a
likelihood ratio test is used in which the full model is given by equation
\ref{eq:dist} and the reduced model is given by a negative binomial
distribution.  If PIHM is not significant in the system, the resulting likelihood
ratio statistic should approximately follow a $\chi^2$ distribution with two degrees of freedom.

\subsection*{Evaluating the Adjei and Likelihood Methods}

\emph{Question 1: Can we detect PIHM?}

We tested the ability of the Adjei and the Likelihood Methods to identify the presence of PIHM on simulated data with known pre-mortality parameters. First, we created a pre-mortality host population by drawing $N_p$ randomly infected hosts from a
negative binomial distribution with parameters $\mu_p$ and $k_p$. Second, we chose values of $a$ and $b$ and calculated the probability of survival
for all $N_p$ hosts using equation \ref{eq:logistic}.  Then, for each host, we drew a random number from a uniform distribution
between 0 and 1 and if the calculated host survival probability was less than this random
number, the host experienced parasite-induced mortality. The parasite distribution in these simulated surviving hosts is equivalent to the observed parasite distribution in a wild host population that has undergone parasite-induced host mortality.

We used these simulated pre-mortality and post-mortality datasets to test the ability
of both methods to correctly determine whether or not PIHM was occurring when the parameters $N_p$, $\mu_p$ and $k_p$ were known.
  For the Adjei Method, $N_p$,
$\mu_p$, and $k_p$ are estimated using the Crofton Method, while $\mu_p$ and
$k_p$ in the Likelihood Method can be estimated jointly with $a$ and $b$ or via the Crofton Method.  Although the parameters $N_p$, $\mu_p$, and $k_p$ are always unknown in real systems, a method that fails under these ideal simulation conditions will certainly also fail using less ideal, empirical data.

We used three different values of $\mu_p$ (10, 50, 100) and for each $\mu_p$ we examined three different survival functions that had graduate, moderate, and steep decreases in the host survival with increasing parasite intensity (Figure \ref{fig:question1}A).  For a given $\mu_p$, each survival function had the same $LD_{50}$ ([$\mu_p = 10$, $LD_{50} = 7.39$], [$\mu_p = 50$, $LD_{50} = 35.57$], [$\mu_p = 100$, $LD_{50}= 77.3$]),  but different values of $a$ and $b$.  We examined each $\mu_p$-survival function pair at  three levels of parasite
aggregation, $k_p = 0.1$, 0.5, and 1 --- realistic values of parasite aggregation in natural populations \citep{Shaw1998}.  For each of these parameter
combinations we simulated 150 datasets and tested the probability of each method correctly identifying PIHM in the post-mortality dataset (power) and incorrectly identifying PIHM in the pre-mortality dataset (Type I error).  For each method, we used a likelihood ratio test to determine whether the full model with PIHM provided a significantly better fit than the reduced model without PIHM at significance level of 0.05.  We tested each parameter combinations for pre-mortality population sizes of $N_p$ = [50, 100, 200, 300, 400, 500]. $N_p$ is not technically the sample size on which the methods are being
tested on the post-mortality data because PIHM reduces $N_p$ for each simulated
dataset.  We therefore used the average number of surviving hosts over all 150 simulations for a given parameter combination as our measure of sample size in the power simulations.\\

\noindent
\emph{Question 2: Can we estimate fatal parasite intensity and the host survival function?}

To compare the ability of the Adjei Method and the Likelihood Method to recover the $LD_{50}$ and the parameters $a$ and $b$ of the survival function, we used the same simulation procedure and parameter combinations described above. For each parameter
combination we simulated 150 datasets, estimated $a$, $b$, and $LD_{50}$ and calculated the standardized bias and
precision \citep{Walther2005} for these estimates over pre-mortality host population sizes of $N_p$ = [300, 500, 1000, 2000, 5000, 7500,
10000]. We used the average number of surviving hosts over all 150 simulations for a given parameter combination as our measure of sample size.  Because parameters $a$ and $b$ showed similar patterns of bias and precision, we only show the results for $a$.

\subsection*{Efficacy of the Likelihood Method with unknown pre-mortality parameters}

In the final simulation, we test the ability of the Likelihood Method to correctly identify PIHM and estimate $LD_{50}$ when the pre-mortality parameters are unknown. The previous simulations showed that the Likelihood Method effectively identified PIHM when $\mu_p$ and $k_p$ were known with values of 10 and 1, respectively.  In the best-case scenario where a host-parasite system has these these parameters, we test the power of the Likelihood Method to identify PIHM for gradual, moderate and steep survival functions when the pre-mortality parameters also needed to be estimated.  We perform 500 simulations over a range of different samples sizes following the simulation procedure described above.

\subsection*{Application to real data}

We tested the ability of the Adjei Method and the Likelihood Method to identify
PIHM in 6 host-parasite datasets given in \cite{Crofton1971a} and 4 datasets
given in \cite{Adjei1986} (Table \ref{table:pihm}). \citeauthor{Crofton1971a} analyzed infection patterns in the snail \emph{Gammarus pulex} infected with the
acanthocephalan \emph{Polmorphus minutus}. \citeauthor{Adjei1986} analyzed males and females of two species of lizard fish \emph{Saurida tumbil} and
\emph{S. undosquamis} that were infected by the cestode
\emph{Callitetrarhynchus gracilis}.

In both earlier studies, the authors reported PIHM in some of the datasets and we test whether the Adjei
Method and/or the Likelihood Method also predicted PIHM. For the 6 datasets from
\cite{Crofton1971a}, we truncated the data at 4 parasites, applied the Crofton
Method to estimate the pre-mortality distribution, and then ran the Likelihood
Method and Adjei Method using these pre-mortality parameters.  For the
\cite{Adjei1986} datasets, we followed the same procedure as the authors and
first truncated the data at 2 parasites and then fit the Crofton Method for the
female fish of both species.  Then, following the original authors' methods, we parameterized the male pre-mortality
distributions for each species with the results from the females.  Finally, we
applied the Adjei Method and the Likelihood Method to determine whether or not
PIHM was significant for these species and compared our results to those given by the authors.  All fitting to data was done with the code provided in \emph{SI} 4.

\section*{Results}

\subsection*{Question 1: Detecting presence of PIHM}

The power of the Adjei Method to detect PIHM in a
system was close to unity for larger sample sizes and tended to
decrease as sample size decreased (Figure \ref{fig:question1}B; \emph{SI} 3 Figs 1-3).  The Likelihood Method had a power close to
unity for all parameter combinations and sample sizes considered.  With gradual
survival functions, the power of the Likelihood Method decreased slightly for small samples sizes (Fig. \ref{fig:question1}B, \emph{SI} 3 Figs 1-3).

The Adjei Method showed highly inflated Type I error rates (i.e. falsely detected
PIHM) for all parameter combinations that we
considered (Fig. \ref{fig:question1}A; \emph{SI} 3 Figs 1-3).  This method also showed the unintuitive pattern of Type I error
rate decreasing as sample size decreased.  This pattern was due to the issue of
binning discussed in the \emph{Introduction} and \emph{SI} 2. For small samples sizes, the
applicability of the Adjei Method is compromised without binning the observed
data in some way.  In contrast, the Likelihood Method showed a Type I
error rate at or near the pre-set level of 0.05 for all parameter combinations
and sample sizes considered (Fig. \ref{fig:question1}A; \emph{SI} 3 Figs 1-3).

\subsection*{Question 2: Estimating the $LD_{50}$ and survival function}

The Likelihood Method gave asymptotically unbiased estimates of the $LD_{50}$
for all combinations of parameters examined in this study (Fig. \ref{fig:question2}, \emph{SI} 3 Figs 4-6).  Even for
the smallest sample sizes we considered, the Likelihood Method's estimate of $LD_{50}$
was largely unbiased, with small biases occurring for host survival functions
that were gradual. The precision of the Likelihood Method's $LD_{50}$ estimates decreased
(increasing coefficient of variation) as sample size decreased for all
parameter combinations we examined (Fig \ref{fig:question2}, \emph{SI} 3 Figs 4-6).

The Adjei Method produced biased estimates of the $LD_{50}$ across nearly all parameter combinations (Fig \ref{fig:question2}, \emph{SI} 3 Figs 4-6)).  For $\mu_p = 10$, the $LD_{50}$
estimates from the Adjei Method were largely unbiased for large samples sizes, but as $\mu_p$ increased, the Adjei Method
produced biased estimates of $LD_{50}$ across all sample sizes, with bias
increasing as sample size decreased (Figure \ref{fig:question2}, SI2 Fig 4-6). The $LD_{50}$ estimates from the Adjei
Method showed large decreases in precision with the steepest survival function across all values of $\mu_p$ (Figure \ref{fig:question2}, SI2 Fig 4-6).

In terms of the host survival function, the Likelihood Method gave unbiased estimates of survival function parameters when sample sizes were large, however as sample size decreased these estimates became severely biased (Fig. \ref{fig:question2}, \emph{SI} Fig 7 - 9) The Adjei Method produced
biased estimates of the host survival function across all sample sizes, with the bias
consistently being larger when the survival function was steeper and $\mu_p$ was larger (Fig \ref{fig:question2}, \emph{SI} 3 Figs 7-9).

\subsection*{Detecting PIHM with unknown pre-mortality parameters}

When all pre-mortality parameters were jointly estimated, the Likelihood Method had a power of greater than 0.8 when the survival function was moderate and steep for host sample sizes of 424 and 83 respectively (Figure \ref{fig:real_power}).  When the host survival function was gradual, the Likelihood Method never had a power greater than 0.8 for any post-mortality samples sizes we considered.

\subsection*{Application to real data}

Of the 10 datasets we considered, the previous authors visually detected PIHM
in 7 of them (Table \ref{table:pihm}).  The Likelihood Method parameterized
from the pre-mortality parameters of the Crofton Method detected significant
PIHM in 6 of these 7 datasets at a significance level of 0.05.  The only
dataset in which the Likelihood Method did not detect a significant effect of PIHM was the Adjei dataset
for female \emph{S. tumbil}.  For this dataset there was a marginally significant effect
of PIHM ($\chi^2_{df=2} = 5.34; p = 0.069$). The Adjei Method detected PIHM in 9 of the 10 datasets (Table \ref{table:pihm}), consistent with our simulation results that the Adjei Method has a high Type I error rate.

\section*{Discussion}


Quantifying the impact of parasitism on wild host populations is critical for managing wildlife populations and understanding parasite transmission. Ideally the relationship between
infection intensity and host survival would be measured experimentally, but for
logistical and ethical reasons, this is often impossible \citep{McCallum2000a}.
Looking for evidence of mortality in parasite distribution data requires the
least amount of information, but is notoriously difficult to implement. The
methodological flaws in the Adjei Method limit its utility, so here we propose
an alternative, likelihood-based, method to estimate host survival and the
$LD_{50}$ from observed parasite intensity data.  This
method is a significant improvement over the previous methods because it requires fewer parameters,
provides a statistical decision rule for identifying PIHM and does not require
any data manipulation.

Using simulated data, we found that the Likelihood Method always out performed the Adjei Method. For simply detecting the presence of PIHM, the Likelihood
Method was both more powerful and had fewer false detection events (Type I
errors).  When both methods were applied to published datasets previously used
in PIHM analyses, the Adjei Method tended to detect PIHM where it had not previously been
reported, consistent with the high Type I error rate observed in our
simulations. The Likelihood Method was also more precise and less
biased in calculations of both the parasite $LD_{50}$ and host survival curve over the parameter values we considered.
However, while only the Likelihood Method produced precise and unbiased $LD_{50}$
estimates, neither method could provide unbiased estimates of the host survival
function at realistic sample sizes.  These simulations demonstrate that
the Likelihood Method is more powerful and precise than the previously propose Adjei Method.

Although superior to the Adjei Method, the Likelihood Method is not universally applicable to real data.  Our simulations showed when the when pre-
mortality parameters were estimated directly, the Likelihood Method needed at
least 83-424 samples to have 80\% power and for steep to moderate survival functions and even
more as the survival function became more gradual. While some of these sample sizes are reasonable for hosts such as invertebrates or small fish, even the smallest sample sizes are completely
unfeasible for many vertebrates, particularly the species of conservation
concern where addressing the impact of parasitism would be most important. An
even larger sample size would be required to identify PIHM when parasites are highly aggregated, mean infection intensity is
high, or parasite prevalence is low, all of which are
common in many parasitic helminths.  Moreover, our results are in agreement with previous work that has shown that as host-survival functions become progressively more linear, PIHM becomes all but impossible to detect \citep{Lanciani1989}.  This result, however, does not preclude the use of this method as non-linear survival functions
are not uncommon in empirical host-parasite systems \citep{Benesh2011}.  Finally,
while linear functions make PIHM undetectable, at the other extreme, steep,
non-linear survival curves produce severely biased estimates of the survival
function. Give the interaction between all of these different factors, the
Likelihood Method is probably limited to detecting PIHM in systems where greater than 100 hosts can be collected, parasites are
common and only moderately aggregated, and substantial host mortality occurs at relatively low parasite intensity.

While we have improved on the existing methods for quantifying
PIHM from parasite intensity data, all such methods require several
fundamental, and potentially problematic assumptions.  Nearly all current methods derive from \cite{Crofton1971a} \citep[but see][]{Ferguson2011} and assume that, prior to any PIHM, parasites are distributed in the host population following a
negative binomial distribution. But, it is fundamentally impossible to know
what the pre-mortality parasite distribution was in a wild host population and
it is widely recognized that different processes can lead to a variety of
parasite distributions in hosts \citep{Anderson1982a, Duerr2003}. However, the negative binomial is extremely
flexible and there is substantial empirical and theoretical evidence to support
the assumption that, prior to any PIHM, parasite distributions can be fit by a negative binomial distribution \citep{Shaw1995,Shaw1998,Wilson2002}.

Unfortunately, this
flexibility in the distribution may also reduce our ability to detect PIHM. If
a negative binomial can be fit to the observed post-mortality parasite
distribution then, regardless of how lethal the parasite was, it will be
impossible to detect PIHM because there is no need for a more complex model.
Most observed parasite distributions are well fit by the negative binomial distribution \citep{Shaw1998}, suggesting
that systems where these methods are applicable may be more the exception than
the rule.  Furthermore, even when truncation of the negative binomial distribution is detected, it may be caused by other
processes such as within host density dependence, age dependent variation in host
resistance and/or heterogeneous infection rates \citep{McCallum2000a,Anderson1982a,Rousset1996}.  This means that in the event
that PIHM is detected, it may actually not be the result of PIHM.

Given these numerous caveats, is there a place in parasitology for methods that
estimate PIHM from intensity data?  We are in agreement with
\cite{Lester1984} that, at the very least, methods for estimating PIHM can
provide preliminary insight into whether or not PIHM is worth further
exploration.  However, we stress that these methods should only be used as an
exploratory tool when assessing the role of PIHM in a system and potential
users should critically evaluate whether they think they have a large enough
sample size and an appropriate host survival function/post-mortality distribution for the methods developed
in this paper to be applicable.  Even if they are applicable, inferring PIHM
from distributional data is no substitute for field experiments
and an in depth understanding of the natural history of the host-parasite
system under consideration.

\section*{Acknowledgments}

TODO


\singlespacing
\bibliographystyle{/Users/mqwilber/Dropbox/Documents/Bibformats/ecology_letters.bst}
\bibliography{/Users/mqwilber/Dropbox/Documents/Bibfiles/Projects_and_Permits-parasite_host_mortality}

\newpage

\renewcommand{\arraystretch}{1.2}

\begin{table}

    \caption{Definition of parameters and functions used in the main text}
    \begin{tabular}{c c}
    \hline
    Parameter & Definition \\
    \hline\hline
    $\mu_p$ & Pre-mortality mean parasite intensity \\
    $k_p$   & Pre-mortality parasite aggregation \\
    $N_p$   & Pre-mortality host population size \\
    $x$     & Number of parasites in a given host \\
    $g(x; \mu_p, k_p)$ & Pre-mortality negative binomial distribution parasites distribution \\
    $a$ & Parameter of the logistic host survival function \\
    $b$ & Parameter of the logistic host survival function \\
    $h(x; a, b)$ & The host survival function \\
    $LD_{50}$ & $\exp(a / b)$, Parasite intensity at which 50\% of hosts die \\

    \end{tabular}
    \label{tab:params}
\end{table}

\renewcommand{\arraystretch}{1.2}
\begin{sidewaystable}

    \caption{Comparison of the PIHM predictions of previously used host-parasite datasets to those given by the Adjei Method and the Likelihood Method. The first column specifies the identity of the dataset, the second column specifies whether or not the author indicated that PIHM was occurring in the system, the third column indicates whether or not the Likelihood Method with pre-mortality parameters estimated from the Crofton Method detects significant PIHM, and the final column indicates whether the Adjei Method with pre-mortality parameters estimated from the Crofton Method detects PIHM.  If a method detected significant PIHM the predicted $LD_{50}$ is given in parentheses}

    \centering
    \begin{tabular}{l  p{3cm} p{3cm} l}

    \hline\hline
    Data Set (sample size) & Author detected PIHM? & Likelihood Method?  & Adjei Method? \\

    \hline
    Crofton, Station 1 ($n=538$) & Yes & Yes (7.27) & Yes (9.33) \\
    Crofton, Station 2 ($n=507$) & Yes & Yes (6.92) &  Yes (14.95)\\
    Crofton, Station 3 ($n=633$) & Yes & Yes (5.93) &  Yes (5.98) \\
    Crofton, Station 4 ($n=486$) & No & No &  Yes (7.99) \\
    Crofton, Station 5 ($n=276$) & No & No & Yes (10.58) \\
    Crofton, Station 6 ($n=191$) & No & No & No \\
    Adjei, \emph{S. tumbil} female ($n=446$) & Yes (5.7) & No & Yes (6.37) \\
    Adjei, \emph{S. tumbil} male ($n=452$) & Yes (3.4) & Yes (3.42) & Yes (3.66)  \\
    Adjei, \emph{S. undosquamis} female ($n=2573$) & Yes (3.2) & Yes (3.04) & Yes (3.11) \\
    Adjei, \emph{S. undosquamis} male ($n=2440$) & Yes (1.8) & Yes (1.83) & Yes (1.78) \\


    \end{tabular}
    \label{table:pihm}

\end{sidewaystable}

\begin{figure}

\begin{tikzpicture}


    \node at (0, 0) (pic) {\includegraphics[width=\textwidth]{/Users/mqwilber/Repos/parasite_mortality/results/figure1_partII_for_manuscript50}};

    \node[above=0.1cm] at (pic.north) (concept) {\includegraphics[width=0.6\textwidth]{/Users/mqwilber/Repos/parasite_mortality/results/figure1_partI_for_manuscript}};

    \draw[->] (0, 4.2)--(-1.5, 3.1);
    \draw[->] (0, 4.2)--(1.5, 3.1);

    \node at (-3.2, 4.9) {A.};
    \node at (-6.5, 2.7) {B.};
    \node at (1.4, 2.7) {C.};

\end{tikzpicture}

\caption{A) Five potential shapes for a host-survival functions. In the simulations we used a gradual survival function (dotted line), and moderate survival function (dashed line), and a steep survival function (solid line). The linear and immediate survival functions represent two potential extremes that we do not include in the simulations. For each of these survival functions and the parameter combinations described in the main text, we tested the Type I Error and Power of the Likelihood Method and Adjei Method. B) Gives the Type I Error of each method over a range of pre-mortality sample sizes with a pre-mortality mean parasite intensity ($\mu_p$) of 50 and pre-mortality parasite aggregation ($k_p$) at 0.5. The red line shows the pre-set significance level of 0.05. C) Gives the Power of each method for detecting PIHM over a range of post-mortality sample sizes for $\mu_p = 50$ and $k_p = 0.5$.  In general, the Likelihood Method has higher power and lower Type I Error than the Adjei Method.  See the SI 3 Fig 1 - 3 for Type I Error and Power results for all parameter combinations.}

\label{fig:question1}

\end{figure}

\begin{figure}
    \includegraphics[width=\textwidth]{/Users/mqwilber/Repos/parasite_mortality/results/figure2_for_manuscript.pdf}

    \caption{Bias and precision (coefficient of variation) for the Likelihood Method and Adjei Method estimates of the $a$ parameter and the $LD_{50}$ of the host survival function based on simulated PIHM data over a range of post-mortality sample sizes. The pre-mortality parameters for this simulation were $\mu_p = 50$ and $k_p = 0.5$.  The figure shows the simulations for three different host survival functions (gradual, moderate, and steep), each with the same $LD_{50}$.  Bias and precision results of $LD_{50}$ and $a$ for all other parameter combinations can be found in SI 3 Fig 4 - 9.}

    \label{fig:question2}

\end{figure}

\begin{figure}
    \includegraphics[width=\textwidth]{/Users/mqwilber/Repos/parasite_mortality/results/figure3_for_manuscript.pdf}

    \caption{The power of the Likelihood Method to detect PIHM for gradual, moderate, and steep survival functions when all four parameters $\mu_p$, $k_p$, $a$, and $b$ were jointly estimated. The curves were generated from 500 simulations for 10 pre-mortality population sizes, $N_p$.}

    \label{fig:real_power}

\end{figure}


\end{document}

